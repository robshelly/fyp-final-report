\section{Semester One Summary}
Semester one consisted of the research phase of this project. As part of the research phase, the technical feasibility of the project was assessed by first defining the following three \textit{Proof of Concepts}:


\begin{POC}
	\item \label{poc1} \textbf{Validation:} How will a restoration be validated? What criteria are needed for a backup to be deemed successful?
	\item \label{poc2} \textbf{User Rules:} How will a user use the system? Can the implementation of a backup restoration be abstracted from the user by means of a user friendly frontend?
	\item \label{poc3} \textbf{Security:} How will the system deal with encrypted backups? Can the Jenkins server safely handle credentials and decrypt backups in order to perform the test restoration?
\end{POC}

\subsection{\ref{poc1} Validation}
  This POC was tested by creating a Jenkins job which restores a backup file to the DBMS and performs a full read of the database.

	\subsubsection{Work Completed}
	In order to prove that validation can be achieved via Jenkins, the following infrastructure was deployed to AWS:
	\begin{itemize}
		\item \textbf{Backup Server:} An EC2 instance on which a backup file was saved.
		\item \textbf{Restoration Server:} An EC2 instance with the MongoDB DBMS installed and running.
		\item \textbf{Jenkins Server.}
	\end{itemize}
	A Jenkins job was then configured to execute the following tasks:
	\begin{enumerate}
		\item Copy backup file from \textit{backup} server to \textit{restoration} server (sample MongoDB data used \citep{mongo});
		\item Import the backup file into MongoDB on \textit{restoration} server;
		\item Run a \textit{findAll} command on the MongoDB database to retrieve all entries;
		\item Print the entries to Jenkins console;
	\end{enumerate}

	\subsubsection{Results}
  The POC showed backups can be successfully transported to and imported into a remote server. Once imported, a successful \textit{findAll} command proves that the data is still readable. The POC also demonstrated that the SSH credentials for the various servers could be stored in Jenkins and used securely into order to connect to them.
  
\subsection{\ref{poc2} User Rules}
	This POC was tested by creating a basic frontend with React which uses Jenkins API to trigger an arbitrary Jenkins Job.
	
	\subsubsection{Work Completed}
  The following tools were utilised to create a web interface for Jenkins.
  \begin{itemize}
    \item A basic web app was created with ReactJS, consisting of a simple form to input parameters;
    \item A parameterised Jenkins job was;
    \item The NodeJS wrapper for the Jenkins API, which allows Jenkins API calls to be made with NodeJS.
  \end{itemize}
  When the web form was completed and submitted, the web app triggered the parameterised job (including the parameters input to the form) via the API.

  \subsubsection{Results}
  The POC demonstrated that implementation of Jenkins jobs can be abstracted from the user therefore allowing the system to present the user with a simple form to complete in order to commence an automated restoration.

\subsection{\ref{poc3} Security}
  This POC was tested using the same AWS infrastructure as \hyperref[poc1]{POC1}. In this instance, the backup file had been encrypted with a GPG.

  \subsubsection{Work Completed}
	In this instance the Jenkins job copied the encrypted backup to the \textit{restoration} server. The backup was then decrypted using GPG and a private key on the \textit{restoration} server and the contents printed in order to verify decryption. 
	
  \subsubsection{Results}
  This POC demonstrated that the system, in its current architecture, can successfully copy an encrypted backup file to the \textit{restoration} server where it is decrypted. It also demonstrated that the GPG private keys can be securely stored and used by Jenkins.

  \subsection{Prototype}
  As a final task, the three POCs were integrated for the purpose of creating a prototype. The prototype consisted of a single form on a web app which allowed the user to input the IP address or DNS and filename of a encrypted backup file. When the form was submitted, it triggered a Jenkins job which moved the file to a restoration server where it decrypted the file. It then imported the file into a DBMS and performed a full read of the database, thereby proving that the backup file was valid.

