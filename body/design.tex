\section{Design}
	\subsection{System Architecture Overview}
		The system will comprise of three main components:
		\begin{itemize}
			\item Management Server
			\item User Interface
			\item Disposable instances/containers
		\end{itemize}
		The system will also use existing ECS instances i.e  where backups are stored. Depending on the user of the system there may be multiple backup servers in different location (such as AWS regions) or for different data types (relational and non-relational databases).
		
		\begin{figure}[H]
			\setlength{\belowcaptionskip}{15pt plus 3pt minus 2pt}
			\caption{Diagram of System Architecture}
			\centering
			\includegraphics[scale=0.5,keepaspectratio]{system-architecture}
			\label{fig:diagram}
		\end{figure}
		
		\noindent \textbf{Management Server:} This will be a small low cost AWS instance on which the Jenkins automation server will be installed. The majority of the system's functionality will be carried out and/or orchestrated by this server. Jenkins jobs will copy the backups from their location to a disposable instance and implement the necessary steps to validate them such as importing and reading.
		
		\noindent \textbf{User Interface:} This will provide a simple user interface (UI) for the system, implemented as a simple web app, hosted on AWS. It will allow users with little knowledge of Jenkins and AWS to perform backup restoration checks by adding a layer of abstraction. Users will be able to run restorations by providing the parameters such as the backup file and its location. The UI will utilise the Jenkins API to execute the restoration with the parameters provided.
		
		\noindent\textbf{Disposable Instances:} Disposable infrastructure will be used to perform the restoration. This will consist of EC2 instances running the necessary DBMS to perform the restoration. They can also be destroyed afterwards, destroying the decrypted backup, data and therefore maintaining confidentiality.

	\subsection{Formal Modelling}
	\subsubsection{Sequence Diagrams}
		The main function of the systems have been demonstrated below in sequence diagrams. \autoref{fig:seq-run-restore} shows the process of running a single backup restore. This involves a user manually triggering a restoration using the web interface. This triggers a Jenkins job which automates the remaining steps:
		\begin{enumerate}
			\item Backup copied to \textit{restoration} server;
			\item Backup decrypted;
			\item Backup import into DBMS;
			\item Data read from DB;
		\end{enumerate}
		Upon completion the \textit{restoration} server is terminated.

		\autoref{fig:seq-schedule-restore} shows the process of a scheduling regular backup restoration tests. Again, this is triggered by a user from the web interface. The web interface will pass the JSON or XML configuration for a job to the Jenkins server. The server verifies the backup server exists before creating the job.
		
		\autoref{fig:seq-delete-restore} Show the process of deleting an existing scheduled job. The user triggers this process from the web interface. This sends a \textit{delete} command to the Jenkins server via the API to remove the schedule job. The status of the command, indicating a successful or failed restore, is returned to the user.
		
		\begin{figure}[H]
			\setlength{\belowcaptionskip}{15pt plus 3pt minus 2pt}
			\caption{Run Restore}
			\centering
			\includegraphics[width=\textwidth,keepaspectratio]{sequence-diagram-run-restore}
			\label{fig:seq-run-restore}
		\end{figure}
		
		\begin{figure}[H]
			\setlength{\belowcaptionskip}{15pt plus 3pt minus 2pt}
			\caption{Schedule Regular Restore}
			\centering
			%\includegraphics[width=\textwidth,height=\textheight,keepaspectratio]{diagram}
			\includegraphics[width=\textwidth,keepaspectratio]{sequence-diagram-schedule-restore}
			\label{fig:seq-schedule-restore}
		\end{figure}
		
		\begin{figure}[H]
			\setlength{\belowcaptionskip}{15pt plus 3pt minus 2pt}
			\caption{Delete Scheduled Restore}
			\centering
			%\includegraphics[width=\textwidth,height=\textheight,keepaspectratio]{diagram}
			\includegraphics[width=\textwidth,keepaspectratio]{sequence-diagram-delete-schedule}
			\label{fig:seq-delete-restore}
		\end{figure}
	
	\subsubsection{User Stories} \label{user-stories}
   
		User stories are provided in \autoref{table:user-stories}. Two types of system users and their privileges are described below;
		
		\textbf{Managers} control security aspects of the system:
		\begin{itemize}
			\item Add and remove regular users;
			\item Manage SSH keys and decryption keys.
		\end{itemize}
		\textbf{Regular Users} will perform the day-to-day tasks of the system:
		\begin{itemize}
			\item Perform restorations;
			\item Schedule restorations;
			\item View restoration results;
		\end{itemize}
		
		\begin{table}[H]
			\small
			\centering
			\setlength{\belowcaptionskip}{15pt plus 3pt minus 2pt}
			\caption{User Stories}
                         
			\begin{tabular}{|l|l|p{0.39\linewidth}|p{0.39\linewidth}|} \hline
				\textbf{\#} & \textbf{As a} & \textbf{I want to be able to} & \textbf{so that} \\ \hline
				US1 & manager & implement a user system & I control who can run backup restores \\ \hline
				US1.1 & manager & add my team members to the system & they can run backup restores \\ \hline
				US1.2 & manager & remove users from the system & former team members no longer have access \\ \hline
				US2 & manager & add and control sensitive information within the system & I can implement a security policy \\ \hline
				US2.2 & manager & securely store credentials within the system & they don't need to be entered every time a restore is executed \\ \hline
				US2.3 & manager & add SSH keys for backup server & the system has secure access backup server \\ \hline
				US2.4 & manager & add decryption keys for backups & encrypted backups can be decrypted for testing \\ \hline
				US2.5 & manager & delete SSH keys & expired/outdated credentials are no longer stored \\ \hline
				US2.6 & manager & delete decryption keys & expired/outdated credentials are no longer stored \\ \hline
				US3 & manager & execute all same tasks as a regular user & I don't need a second set of credentials to run restores myself \\ \hline
				US4 & user & login & I can run restores \\ \hline
				US5 & user & logout & I avoid potential unauthorised access \\ \hline
				US6 & user & run a test restoration of a backup & I can verify that the backup exists, is a valid file, and is readable \\ \hline
				US6.1 & user & run a test by filling out a simple form with basic parameters (location, filename) of the backup to test & I can easily run a restore of a specific backup without needing to worry about the implementation \\ \hline
				US6.2 & user & view the current status of a running restoration & I can review the progress of long running restores \\ \hline
				US6.3 & user & check if a backup failed or succeeded & I can immediately investigate any failed backups \\ \hline
				US7 & user & create a schedule of automated restores for a given backup & I don't have to manually execute them myself on a regular basis \\ \hline
				US7.1 & user & choose the frequency of automated restores within a schedule, from daily through weekly to monthly & I control how often different backups are tested \\ \hline
				US7.2 & user & check if an automated restoration has started & I can verify my schedule is working correctly \\ \hline
				US7.3 & user & check the results of an automated restore & I can immediately investigate any failed backups \\ \hline
				US7.4 & user & view the all past results of an automated restore schedule & view the consistency of my backups success \\ \hline
				US7.5 & user & modify a scheduled restore & I can change the frequency of a scheduled restore \\ \hline
				US7.6 & user & the parameters of a schedule & any changes to the backups, such as location, will be reflect in the restoration schedule \\ \hline
				US7.7 & user & delete regularly scheduled restores & old backups/deleted backups are no longer tested \\ \hline 
				US8 & user & view feedback of a failed restore & I might gain an insight into the fault in the backup \\ \hline
				US9 & user & notified when a restoration fails & silent, unnoticed fails are avoided \\ \hline
			\end{tabular}
			\label{table:user-stories}
		\end{table}
		

        
	\subsection{Front End Design}
	\subsubsection{Wireframes}
	
		Wireframes for the frontend are shown below. 
        
        \autoref{fig:homepage} shows the homepage. It includes the following components:
		\begin{itemize}
			\item Form for running a restore;
			\item Form for creating a restore schedule;
			\item List of schedules (including links).
		\end{itemize}
		\autoref{fig:scheduled} shows the details and past results of a scheduled restore.
		
		
		\begin{figure}[H]
			\setlength{\belowcaptionskip}{15pt plus 3pt minus 2pt}
			\caption{Homepage}
			\centering
			\includegraphics[width=\textwidth,keepaspectratio]{wireframe-homepage}
			\label{fig:homepage}
		\end{figure}
		
		\begin{figure}[H]
			\setlength{\belowcaptionskip}{15pt plus 3pt minus 2pt}
			\caption{Scheduled Restore}
			\centering
			\includegraphics[width=\textwidth,keepaspectratio]{wireframe-scheduled-restore}
			\label{fig:scheduled}
		\end{figure}

	\subsection{Enterprise Level Consideration}
  As this project is being designed for the product owners to potentially deploy it to test production database backups, there are a number of key aspects which should be considered. This is to ensure that the product does not just achieved its main goals but does so in a secure and useable manner for the product owners.
  
  \subsubsection{Security}
  Security is a key aspects of this project. In fact , due to the importance of security it has been defined as one of the key aims of the project (\hyperref[ao4]{AO4}). The security of this system shall by determined by two main areas.
  
  \subsubsubsection{Storage of Private Keys}
  This pertains to the storage of the SSH private keys for accessing backup servers and the GPG private keys used to decrypt the backups. Safe storage of such keys is paramount to the project as exposing these keys would compromise production data. Accordingly, during the research phase of the project, \hyperref[poc3]{POC3} was concerned with the investigation of this area.
  In this POC, Jenkins Credentials Plugin was used to store the credentials. A prototype was then used to demonstrate that the private keys could be used to implement a backup restoration while maintaining the confidentiality of the keys.
  
  \subsubsubsection{User Authentication}
  Users of the system will not be able to retrieve any sensitive data through its use, rather they will only have the ability to schedule and run backup restorations. Restorations do not affect the existing backup files or servers as the test is completed on a disposable server in AWS. Also restorations are idempotent, meaning running more than one test on a given backup won't affect the results.
  
  Therefore, malicious users of the system could not affect data confidentiality. However, they could incur increased AWS costs by running and scheduling extra restorations. For that reason access to the UI should be restricted. Accordingly,  a user authentication system has been implemented. This system currently consists of a single \textit{admin} user, created during the Docker image build with credentials provided. For a minimum viable product it has been decided that this is sufficient. However, plans for future development included a full User system, allowing t\textit{admin} user to add other uses to the systems. This would allow a team manager to provide each team member with their own set of credentials to use the system.
  
  \subsubsection{Deployment}
  During discussion with the product owners it was decided that the deployment on installation of the final product would be a simple task, requiring little or no configuration of the user. For this reason it was decided that the final product should be delivered as a Docker image.
  
  However, the final system consists of two servers; the management server (Jenkins) and a web server (UI). This means two services are required to run within the Docker image. However, this is not the recommended practice for Docker images \citep{docker2}. A container should be run as a single process which if terminated will terminate the container. For this reason the final product should be package as two images. These images can still be easily run with any configuration required. They will run in separate containers but are preconfigured to be able to communicate.
  
  This method of deploying systems as a network of containers is in fact a common practice in the industry, leading to the development of container management platforms such as Kubernetes \citep{kubernetes}. Accordingly, OpenShift, Red Hat's own container management platform has been considered as the suitable deployment location for the final product.
  
  \subsubsection{Performance}
  The use of containers to run the final products play a key role in the performance of the system. Containers are similar to virtual machines allowing them to run on a variety of platforms. However, as containers virtualise software instead of hardware, they are much smaller and faster to run \citep{docker}.
  
  Packaging the final product as Docker images is therefore the suitable solution for implementing a high performance system. The images will contain, and therefore run, only the necessary software to run both the Jenkins server and the node app. Indeed, both the Jenkins server and app server can be easily run on a single device.
  